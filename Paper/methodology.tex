\section{Methodology}
Write something about what we're comparing with, the codes we're
using, and how we set up the simulations.

\subsection{Measured offshore conditions}
\fxnote{Lawrence write this part}

Following the work of Cheung et al \cite{cheung2020large}, we use
measurements of the offshore coastal marine boundary layer, collected
by the Cape Wind meteorological tower in Nantucket Sound, as the basis
for this computational study.  The Cape Wind platform collected wind
measurements at 20m, 41m, and 60m above the mean water level, along
with temperature and barometric measurements from the years 2003-2011.
From the observations, Archer et al.\cite{archer2016predominance}
found that the marine boundary layer is predominantly unstable, with
61\% of conditions classified as unstable, versus 21\% neutral and
18\% stable.  The stratification of the marine boundary layer, as
determined by the Obukhov length, also had a large impact on the wind
speed profile, with flatter, non-logarithmic profiles seen during
unstable conditions.

From the measured distribution of atmospheric stabilities, turbulent
kinetic energies (TKE), and wind speeds from the Cape Wind platform, a
specific set of conditions were chosen as targets for this
computational study.  A summary of the conditions for all stability
classes is given table \ref{tab:CapeWindMeasurements}. In this work we
focus on the three stable atmospheric conditions at wind speeds of
5m/s, 10m/s, and 15m/s, while the neutral and unstable conditions
where studied in previous work \cite{cheung2020large}.

To maintain consistency with the measured data, the turbulence
intensity (TI) is calculated using the TKE as
\begin{equation}
  \textrm{TI} =
  \frac{\sqrt{\frac{2}{3}\times\textrm{TKE}}}{\overline{U}_{horiz}} =
  \frac{\sqrt{\frac{1}{3}\left( \overline{u'u' + v'v' + w'w'}
      \right)}}{\overline{U}_{horiz}}
\end{equation}
The averaged wind speeds were enforced at the measurement height of
20m, and the wind directions...

\begin{table}
\caption{\label{tab:CapeWindMeasurements} Measured conditions at Cape
  Wind.  The stable atmospheric conditions used in this study are
  highlighted in bold below.} \centering
\begin{tabular}{cccc}
  \hline
  Stability    & Wind speed [m/s] & Wind dir [deg] & Turbulence intensity \\
  \hline
  Neutral      & 5                & 225            & 0.055           \\
  Neutral      & 10               & 225            & 0.055           \\
  Neutral      & 15               & 225            & 0.065           \\
  Unstable     & 5                & 315            & 0.080           \\
  Unstable     & 10               & 315            & 0.075           \\
  Unstable     & 15               & 315            & 0.090           \\
  \bf{Stable}  & \bf{5}           & \bf{225}       & \bf{0.045}      \\
  \bf{Stable}  & \bf{10}          & \bf{225}       & \bf{0.050}      \\
  \bf{Stable}  & \bf{15}          & \bf{225}       & \bf{0.060}      \\
\hline
\end{tabular}
\end{table}


\subsection{Computational methodlogy}
Provide a general description of the LES codes that we use.

\subsection{LES formulation}

\subsubsection{Governing equations}

\subsubsection{Lower wall model BC}
\fxnote{Ganesh write this part}\\
What we implemented in terms of the lower boundary condition.

\subsubsection{Nalu-Wind}
\fxnote{Shreyas write this part}\\
Describe Nalu-Wind

\subsubsection{AMR-Wind}
\fxnote{Mike write this part}\\
Describe AMR-Wind

\subsection{Computational set up}
\fxnote{Lawrence write this part}\\
Domains, grids, boundary conditions

\begin{table}
\caption{\label{tab:Computational parameters} LES parameters for stable conditions}
\centering
\begin{tabular}{cccc}
  \hline
  Stability    & Wind speed  & Surface roughness $z_0$ & Temperature gradient \\
  \hline
  Stable       & 5  m/s           & 0.0005 m       & -0.32 K/hr      \\
  Stable       & 10 m/s           & 0.0005 m       & -1.40 K/hr      \\
  Stable       & 15 m/s           & 0.0005 m       & -1.50 K/hr      \\
\hline
\end{tabular}
\end{table}

\subsubsection{Simulation workflow}
\fxnote{Lawrence write this part}\\
Describe the process to hit target conditions.




