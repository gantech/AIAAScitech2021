\section{Methodology}
Write something about what we're comparing with, the codes we're
using, and how we set up the simulations.

\subsection{Measured offshore conditions}
% \fxnote{Lawrence write this part}

Following the work of Cheung et al \cite{cheung2020large}, we use
measurements of the offshore coastal marine boundary layer, collected
by the Cape Wind meteorological tower in Nantucket Sound, as the basis
for this computational study.  The Cape Wind platform collected wind
measurements at 20m, 41m, and 60m above the mean water level, along
with temperature and barometric measurements from the years 2003-2011.
From the observations, Archer et al.\cite{archer2016predominance}
found that the marine boundary layer is predominantly unstable, with
61\% of conditions classified as unstable, versus 21\% neutral and
18\% stable.  The stratification of the marine boundary layer, as
determined by the Obukhov length, also had a large impact on the wind
speed profile, with flatter, non-logarithmic profiles seen during
unstable conditions.

From the measured distribution of atmospheric stabilities, turbulent
kinetic energies (TKE), and wind speeds from the Cape Wind platform at
z=20m, a stable set of conditions were chosen as targets for this
computational study.  A summary of the conditions for all stability
classes is given table \ref{tab:CapeWindMeasurements}. In this work we
focus on the three stable atmospheric conditions at wind speeds of
5m/s, 10m/s, and 15m/s, while the neutral and unstable conditions
where studied in previous work \cite{cheung2020large}.

To maintain consistency with the measured data, the turbulence
intensity (TI) is calculated using the TKE as
\begin{equation}
  \textrm{TI} =
  \frac{\sqrt{\frac{2}{3}\times\textrm{TKE}}}{\overline{U}_{horiz}} =
  \frac{\sqrt{\frac{1}{3}\left( \overline{u'u' + v'v' + w'w'}
      \right)}}{\overline{U}_{horiz}}
\end{equation}
The averaged wind speeds were enforced at the measurement height of
20m, and the applied wind directions were consistent with the
predominant stable direction of 225 degrees southwest.

\begin{table}
\caption{\label{tab:CapeWindMeasurements} Measured conditions at Cape
  Wind.  The stable atmospheric conditions used in this study are
  highlighted in bold below.} \centering
\begin{tabular}{cccc}
  \hline
  Stability    & Wind speed [m/s] & Wind dir [deg] & Turbulence intensity \\
  \hline
  Neutral      & 5                & 225            & 0.055           \\
  Neutral      & 10               & 225            & 0.055           \\
  Neutral      & 15               & 225            & 0.065           \\
  Unstable     & 5                & 315            & 0.080           \\
  Unstable     & 10               & 315            & 0.075           \\
  Unstable     & 15               & 315            & 0.090           \\
  \bf{Stable}  & \bf{5}           & \bf{225}       & \bf{0.045}      \\
  \bf{Stable}  & \bf{10}          & \bf{225}       & \bf{0.050}      \\
  \bf{Stable}  & \bf{15}          & \bf{225}       & \bf{0.060}      \\
\hline
\end{tabular}
\end{table}


\subsection{Computational methodology}
Provide a general description of the LES codes that we use.

\subsection{LES formulation}

\subsubsection{Governing equations}
\fxnote{Shreyas write this part.  Currently placeholder equations below.}\\

The low-Mach number equations:
\begin{align}
  \frac{\partial \rho} {\partial t} + \frac{\partial \rho u_j}{\partial x_j} & = 0, \\
\frac{\partial \rho u_i}{\partial t} + \frac{\partial \rho u_j u_i}{\partial x_j}
+ \frac{\partial P}{\partial x_i} & = \frac{\partial \tau_{ij}}{\partial x_j}
+ \left( \rho - \rho_{\circ} \right) g_i, \\
\frac{\partial \rho h}{\partial t} + \frac{\partial \rho u_j h}{\partial x_j} & =
- \frac{\partial q_j}{\partial x_j} + \frac{\partial P_{th}}{\partial t},
\end{align}

\subsubsection{\label{sec:wallmodelBC}Lower wall model BC}
\fxnote{Ganesh write this part}\\
What we implemented in terms of the lower boundary condition.

\subsubsection{Nalu-Wind}
\fxnote{Shreyas write this part}\\
Describe Nalu-Wind

\subsubsection{AMR-Wind}

\fxnote{Do we want a github link reference for amr-wind? exawind?}\\

As part of a suite of physics codes within the open-source project called ExaWind, 
AMR-Wind enables more efficient and scalable simulations of wind power plants. 
This is because it is built on top of the AMReX software framework which contains 
all of the functionality needed to develop massively parallel, block-structured 
adaptive mesh refinement (AMR) applications \cite{AMReX_JOSS}.
Since AMR-Wind is limited to Cartesian block-structured grids it is specialized
to handle atmospheric boundary layer physics and wind turbine wakes. 
The block-structured grids enable more efficient algorithms such as 
Multi-Level Multi-Grid (MLMG) \cite{AMReX_JOSS}
and are well suited for next generation supercomputers that use GPU's. 

AMR-Wind solves the incompressible Navier-Stokes equations and is generalized to
handle variable density and viscosity. Additionally scalar transport equations can be 
solved such as potential temperature or turbulence models. The discretization in AMR-Wind
is based on the approximate projection method used in IAMR \cite{almgren1998conservative} 
and incflo \cite{sverdrup2018highly}. It is a semi-staggered scheme where the velocity and scalar 
variables are located at cell centers and pressure is located at nodes. Pressure is also staggered
in time so that pressure and the pressure gradient (located at cell centers) are at time $n+1/2$. 
The time discretization is handled with a Crank Nicholson approach where the right hand side 
of the system of equations is evaluated at time $n+1/2$. The advection term is handled explicitly
using an upwind finite-volume method. There are multiple options
within AMR-Wind, but in this paper only the high-order Godunov Piecewise Parabolic 
method (PPM) is used \cite{Colella1984}. The advection terms are projected to face 
centers at $n+1/2$ and are corrected using an exact MAC projection \cite{almgren1998conservative}
which guarantees a divergence free flow. The diffusion terms can be handled explicitly, semi-implicitly, 
or fully implicit and are the discretized using a second-order central difference formula. 
For the simulations in this paper we use a fully implicit scheme as the variable viscosity 
from the eddy viscosity may cause time step restrictions. After the scalar equations
and the momentum equations are advanced in time a nodal projection is used to 
approximately correct the velocity field to make it divergence free. Where the nodal projection
is based on a node centered finite element method \cite{almgren1998conservative}.

%IAMR paper \cite{almgren1998conservative}
%AMReX \cite{AMReX_JOSS}
%incflo \cite{sverdrup2018highly}
%Godunov ppm \cite{Colella1984}

\subsection{Computational setup}
% \fxnote{Domains, grids, boundary conditions}


\begin{table}
\caption{\label{tab:z0tempparam} LES parameters for stable ABL conditions}
\centering
\begin{tabular}{ccccc}
  \hline
  Stability & Wind speed & Surface roughness $z_0$ & Surface
  temperature change & timestep\\
  \hline
  Stable       & 5  m/s           & 0.0005 m       & -0.32 K/hr   & 0.25 sec   \\
  Stable       & 10 m/s           & 0.0005 m       & -1.40 K/hr   & 0.125 sec  \\
  Stable       & 15 m/s           & 0.0005 m       & -1.50 K/hr   & 0.0625 sec \\
\hline
\end{tabular}
\end{table}

The computational methodology for the AMR-Wind and Nalu-Wind LES codes
follows practices similar to the previous offshore ABL study, with
some modifications to handle the stable stratification.  As in
\cite{cheung2020large}, the domain was a square prism geometry, with
the $x$ and $y$ coordinates aligned in the East and North directions,
respectively.  The domain size and mesh requirements were determined
through a numerical grid study.  As discussed in section
\ref{sec:gridstudy}, the horizontal dimensions of 750m $\times$ 750m
were found to be sufficient to capture any large scale structures in
the boundary layer, and in all cases the vertical dimension of 1000m
was also used.  The grid study in section \ref{sec:gridstudy} also
showed that the stable stratification required greater mesh refinement
to capture the smaller turbulent scales, and a uniform cell size of
2.5m was adopted for the offshore LES computations across both codes.

Momentum source terms were included in Nalu-Wind and AMR-Wind to
ensure that the horizontally averaged velocity matched the targeted
wind speed at the z=20 m height.  These source terms are based on the
difference between the desired wind velocity and the instantaneous
horizontally averaged velocity, and are only a function of time and
height z.  Coriolis forcing matching the Cape Wind latitude was
included to capture the effect of wind change with elevation.

\subsubsection{Boundary and initial conditions}
In both horizontal directions of the computational domain, periodic
boundary conditions were applied.  At the lower boundary, we chose to
represent the air/ocean interface using flat boundary with small
amount of surface roughness.  This allowed the wall model discussed in
section \ref{sec:wallmodelBC} to be applied.  Monin-Obukhov similarity
theory was used to determine the velocity and temperature profiles
near the lower surface given a surface roughness height $z_0$ and the
prescribed surface temperature as a function of time.  At the upper
surface of the domain, a potential flow based boundary condition is
applied along with a normal temperature gradient of 0.003 K/m.

The initial temperature profile in all cases was a constant 300K until
the specified inversion height of 650m.  The inversion layer thickness
was 100m, and above this, the temperature linearly increased until it
reached 308.75K at the upper boundary.  The initial mean velocity
profile was uniform throughout the domain with superimposed sinusoidal
velocity perturbations of magnitude 1 m/s to promote the development
of turbulence.

\subsubsection{Determination of surface roughness and prescribed temperature }
To match the measured TI conditions given in table
\ref{tab:CapeWindMeasurements}, the surface roughness and prescribed
surface temperature change were adjusted through an initial
trial-and-error process.  The final surface roughness of $z_0$=0.0005m
used the stable offshore conditions (see table \ref{tab:z0tempparam})
matched the roughness used in the neutral 15m/s and unstable 5m/s and
10m/s cases from the previous study \cite{cheung2020large}.  This
value of surface roughness is consistent with the measurements from
the North Sea \cite{taylor2001dependence}, which found values of $z_0$
ranging from $5 \times 10^{-5}$m to $5\times 10^{-3}$m.

The values of the prescribed temperature change at the ocean surface
is also given in table \ref{tab:CapeWindMeasurements}.  As the wind
speed increases, a larger temperature decrease was used to maintain
similar levels of stable stratification, as expected from
Monin-Obukhov similarity theory.

Once the appropriate surface roughness and prescribed temperature were
determined, and the correct mesh requirements known from the study in
section \ref{sec:gridstudy}, the LES computations for the stable 5m/s,
10m/s, and 15m/s ABL cases could be set up and run.  Each of the cases
was run for 15,000 seconds before collecting statistics for another
5,000 seconds.  To maintain a CFL number less than unity during these
runs, a smaller timestep was used for the higher wind speeds (see
table \ref{tab:z0tempparam}).
