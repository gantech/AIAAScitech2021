\documentclass[conf]{new-aiaa}
%\documentclass[journal]{new-aiaa} for journal papers
\usepackage[utf8]{inputenc}

\usepackage{graphicx}
\usepackage{amsmath}
\usepackage[version=4]{mhchem}
\usepackage{siunitx}
\usepackage{longtable,tabularx}
\setlength\LTleft{0pt} 

\title{Computation and comparison of the stable Northeastern US marine boundary layer}

\author{Lawrence C. Cheung\footnote{Principal member of technical staff, Thermal/Fluid Science \& Engineering, AIAA member}}
\affil{Sandia National Laboratories, Livermore, CA 94550}

\author{
  Shreyas Ananthan\footnote{SOME JOB TITLE, AIAA Member},
  Michael J. Brazell\footnote{Researcher III, High-Performance Algorithms and Complex Fluids, AIAA Member},
  Matthew J. Churchfield\footnote{SOME JOB TITLE, AIAA Member}, \\
  Ganesh Vijayakumar\footnote{Researcher III, Mechanical Engineering, AIAA Member}, and
  Shashank Yellapantula\footnote{SOME JOB TITLE, AIAA Member}
}
\affil{National Renewable Energy Laboratory, Golden CO 80401}

\author{Nathaniel B. deVelder\footnote{Postdoctoral appointee, AIAA member} and
  Alan S. Hsieh.\footnote{Postdoctoral appointee, Wind Energy Technologies Department, AIAA member}}
\affil{Sandia National Laboratories, Albuquerque, NM 87185}

\begin{document}

\maketitle

\begin{abstract}
Put abstract here
\end{abstract}

\section{Nomenclature}

{\renewcommand\arraystretch{1.0}
\noindent\begin{longtable*}{@{}l @{\quad=\quad} l@{}}
$\alpha$  & Wind shear exponent  \\
$L$       & Obukhov length scale \\
$u_\tau$   & Friction velocity    \\
$TI$      & Turbulence intensity 
\end{longtable*}}

\section{Introduction}
\lettrine{P}{ut} the introduction here.

The planned installation of several offshore wind energy plants in the
United States has highlighted the need to properly understand the wind
resource and atmospheric characteristics of the US Atlantic Coast.
Several atmospheric phenomena particular to this region, such as
coastal low-level jets or seasonal Nor’easters, have the potential to
substantially impact the operation and power production of offshore
wind plants.  Of particular interest to the current study is the
atmospheric stability of the marine boundary layer in the Northeastern
US.

Atmospheric stability plays a large role in determining the power
production of wind plants because it directly affects the vertical
distribution of momentum and turbulent energy in the atmospheric
boundary layer (ABL).  The differences between the stable, neutral, or
unstable stratified ABL can lead to large changes in wind speed or
turbulence profiles, and ultimately change the operation of wind
turbines.  Atmospheric stability may also play a role in the formation
of low-level jets \cite{nunalee2014mesoscale} and cause increased
fatigue loads on offshore wind turbines.

Several recent measurement campaigns have provided data to understand
the ABL and wind characteristics for potential offshore wind farms in
the Northeastern US.  Pichugina et al. \cite{pichugina2017properties}
measured the wind profiles and vertical shear profiles in the Gulf of
Maine using a ship-borne lidar approach.  Analysis of measured data
sets from Nantucket Sound by Archer et
al. \cite{archer2016predominance} showed a predominance of low-shear,
unstable conditions at the site.  However, strong seasonal variations
and stratification changes due to diurnal variation were also
observed.

In addition to these measurements, large eddy simulations (LES) have
also been used to study ABL stability characteristics.  Recent work by
Kaul et al. \cite{kaul2020large} has shown that LES computations using
Nalu-Wind can capture the neutrally and convectively unstable onshore
ABL.  Previous simulations by Cheung et al. \cite{cheung2020large}
successfully replicated the unstable and neutral ABL corresponding to
the Cape Wind meteorological tower measurements
\cite{archer2016predominance} and showed the effects of surface
heating on atmospheric stratification.  Additional LES computations
\cite{sullivan2016turbulent} have also previously explored stable
conditions of onshore boundary layers, such as the GABLS1 boundary
layer \cite{beare2006intercomparison}.  However, a complete comparison
including offshore stable ABL conditions has yet to be completed.


\section{Methodology}
Write something about Nalu-Wind and AMR-Wind.

\section{Results}
Write something about Nalu-Wind and AMR-Wind.

\section{Conclusion}
Put conclusion here.

\section*{Appendix}

An Appendix, if needed, should appear before the acknowledgments.

\section*{Acknowledgments}
This research was supported by the Wind Energy Technologies Office of
the US Department of Energy Office of Energy Efficiency and Renewable
Energy.  Sandia National Laboratories is a multimission laboratory
managed and operated by National Technology \& Engineering Solutions
of Sandia, LLC, a wholly owned subsidiary of Honeywell International
Inc., for the U.S. Department of Energy's National Nuclear Security
Administration under contract DE-NA0003525. The views expressed in the
article do not necessarily represent the views of the U.S. Department
of Energy or the United States Government.

% \bibliography{sample}
\bibliography{references}

\end{document}
